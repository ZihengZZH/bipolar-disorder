
Bipolar Disorder (BD) is a serious mental disorder that causes alternating periods of depression and abnormal mood \cite{american2013}. According to the World Health Organisation \cite{world2017}, 3.9 percent of the general population of the United States are affected by BD in their lives. The fourth edition of the \textit{Diagnostic and Statistical Manual of Mental Disorders Text Revision} (DSM-IV-TR) \cite{american2000} includes different subtypes of BD and the episodes are classified into mania, hypo-mania, depression (or remission in some literature), and mixed episodes. Patients often suffer from mood oscillations day to day, which is defined as the rapid cycling of episodes by many clinicians \cite{hilty2006}. BD is also associated with significant mortality risk, with nearly 25 percent of patients attempting suicide and 11 percent of patients completing \cite{prien1990}.

In practice, clinicians identify different manic symptoms in BD during in-person clinical interviews by assessing both verbal and non-verbal indicators of symptoms \cite{hall1995, sobin1997}, such as pitch, volume, gestures and gaze. The Young Mania Rating Scale (YMRS) is one of the most frequently utilized rating scales, and it is based on clinician's evaluation in eleven aspects \cite{young1978}. Assessing manic symptoms is thus time-intensive and subjective with respect to clinicians or other trained raters.

Although some psycho-therapeutic options are available and promising in relapse prevention, only a small proportion of individuals in need actually receive treatment \cite{kazdin2011}. As the treatments are generally sub-optimal or unsatisfactory, treatment refractoriness remains one of the biggest challenges in the treatment of BD \cite{bauer2017}. In addition, Hilty \textit{et al.} \cite{hilty2006} found that the diagnosis of BD usually occurred years after the onset of the disorder, and therefore patients' well-being can be unfavourably impacted because of insufficient recognition or delay in the diagnosis.

Therefore, the automatic recognition of BD could help early detection of relapses and reduce the treatment resistance \cite{bauer2017, cciftcci2018}. Moreover, it could help as a toolkit to assist psychologists during the face-to-face interviews. It could be further deployed to mobile devices to facilitate public access to mental health care \cite{haque2018}. Similar to the evaluation of experts, the automatic solution should rely on ``data" during in-person interviews: visual information, audio signal, and patients' speech. Traditionally, information about the same phenomenon can be obtained from different types of detectors, and the term ``modality" refers to one detector \cite{lahat2015}. Information with three modalities (audio-visual-textual) is captured during the interviews. While one single modality rarely provides complete information of the manic symptoms, each modality brings some added value that cannot be obtained from any of other modalities. The added value is known as \textit{diversity} and the complete information gathered from all available modalities is defined as \textit{complementarity} \cite{lahat2015, mcgurk1976}. Because of \textit{diversity} and \textit{complementarity} of multi-modalities, it is significantly important to take advantage of different modalities to reach the final diagnosis, called \textit{feature-based fusion} \cite{calhoun2008}. 

This work proposes a multi-modal learning framework to classify the mental states of BD patients into three categories - remission, hypo-mania and mania - with the audio-visual-textual data in a recently introduced BD corpus \cite{cciftcci2018}. Different architectures are implemented and evaluated for different modalities. In audio-visual modalities, I present a Multimodal Deep Denoising Autoencoders to learn the joint, frame-level representations and a Fisher Vector encoding to produce the Fisher Vectors for interview sessions. The Paragraph Vector models are utilised to encode transcripts into document embeddings for the textual modality. The early fusion strategy is applied to fuse features from audio-visual and textual modalities before feeding to a Multi-Task Neural Network on the final classification task. The joint representations learnt across modalities are validated by experimental results and the proposed framework achieves the state-of-the-art performance with the unweighted average recall at 0.709 and the accuracy at 0.717.

The contributions of this dissertation can be summarized as follows:
\begin{itemize}
    \item An automatic detection system is developed on the BD corpus, which helps the researchers to provide insights into biological markers within BD and assists psychologists in the BD diagnosis.
    \item In the proposed multimodal framework, different learning architectures are utilized on each of the audio-visual-textual modalities, which manages the discrepancy across modalities, and the learnt multimodal features are fused as the early fusion strategy to reach the final decision, which makes use of the maximal amount of information related to mental disorders.
    \item The framework could be employed to address other similar problems, such as the recognition of other mental disorders or the detection of mental states, both of which require the multimodal fusion.
\end{itemize}


The rest of the report is organized as follows. Section \ref{ch:background} introduces the background knowledge for the following sections. In Section \ref{ch:literature}, the literature in the field of automatic detection mental states is reviewed and the proposed framework is justified. Section \ref{ch:design} presents the design and implementation of the proposed architectures on different modalities respectively. The evaluation and discussion of the framework are covered in Section \ref{ch:evaluation} with conclusions and future work given in Section \ref{ch:summary}.

